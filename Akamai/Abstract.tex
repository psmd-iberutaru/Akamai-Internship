\documentclass[twocolumn]{article}

\usepackage[letterpaper,margin=1in,left=0.5in,right=0.5in]{geometry}
\usepackage{fancyhdr}
\usepackage[plain]{fancyref}
\usepackage{titlesec}
\usepackage{authblk}
\usepackage{setspace}

\usepackage{textcomp}
\usepackage{mathtools}
\usepackage{amsmath}
\usepackage{amssymb}
\usepackage{cancel}
\usepackage{siunitx}
\usepackage{multicol}
\usepackage{pgfplots}
\usepackage{graphicx}

\titleformat{\section}
{\centering\normalfont\scshape}{\thesection}{1em}{}
\titleformat{\subsection}
{\centering\normalfont\scshape}{\thesubsection}{1em}{}
\titleformat{\subsubsection}
{\centering\normalfont\scshape}{\thesubsubsection}{1em}{}

\begin{document}
%% Some preamble stuff.
%%%%%%%%%%%%%%%%%%%%

%% Define the title properties and author information.
%%%%%%%%%%%%%%%%%%%%
\title{Akamai Inquiry \# 2: Rainbow Experiment}

\author[1]{Kenji Emerson}

\affil[1]{\footnotesize UH Hilo, Physics and Astronomy Department}

%% Begin to create the abstract of the article.
%%%%%%%%%%%%%%%%%%%%
\twocolumn[
\begin{@twocolumnfalse}
	\maketitle
	% Ensure that an abstract for the paper is made! This is importiant.
\begin{center}
\begin{minipage}{0.8\textwidth}
	\begin{abstract}

	The meteorological phenomenon of rainbows is caused by water droplets reflecting and refracting sunlight. This phenomenon can be simulated by tiny glass beads on a board with a hand-held light source. The source and cause of the rainbow via glass beads are expected to be similar to that of raindrops in sunlight. Experiments were conducted to compare the known properties of rainbows generated by water droplets and those of the glass beads. We tested the refraction of light through two semicircular glass lens, the glass's index of refraction's dependance on wavelength, and tested to confirm if the geometry of water-droplet rainbows hold true for glass bead rainbows. In the case of two semicircular glass lens, the seperation between the two lens were incompatible with the model of a circular glass bead as it causes unwanted reflection and is an inconclusive test. Using different colored lasers, glass is observed to have a wavelength dependance for its index of refraction, the source of prismiatic effects. The tests of the colored lasers have also provided evidience that the geometry of water-droplet rainbow is the same as glass beads. Through these findings, we find that rainbows caused by tiny glass beads is caused by the same mechnisims as that of water droplets.
	
	\end{abstract}
\end{minipage}
\end{center}
\vspace{1cm}
\end{@twocolumnfalse}
]
%\clearpage % Optional to make a new page.

%% Formally begin to make the article.
%%%%%%%%%%%%%%%%%%%%
\section{First Section}




\end{document}