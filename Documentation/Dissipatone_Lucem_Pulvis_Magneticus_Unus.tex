\documentclass[twocolumn]{article}

\RequirePackage{Polarization_Modeling__research_documentation}

\begin{document}
%% Define the title properties and author information.
%%%%%%%%%%%%%%%%%%%%
\title{Dust Scattering Polarization Modeling \\
\textit{\Large Dissipatone Lucem Pulvis Magneticus Unus}}

\author[1]{Kenji Emerson}
\author[2]{Ramprasad Rao (Mentor)}

\affil[1]{\footnotesize University of Hawaii at Hilo}
\affil[2]{\footnotesize SMA - Smithsoanian Institute}

%% Begin to create the abstract of the article.
%%%%%%%%%%%%%%%%%%%%
\twocolumn[
\begin{@twocolumnfalse}
	\maketitle
	% Ensure that an abstract for the paper is made! This is importiant.
\begin{center}
\begin{minipage}{0.8\textwidth}
	\begin{abstract}

	Abstract!
	
	\end{abstract}
\end{minipage}
\end{center}
\vspace{1cm}
\end{@twocolumnfalse}
]
%\clearpage % Optional to make a new page.

%% Formally begin to make the article.
%%%%%%%%%%%%%%%%%%%%
\section{Introduction}
\label{sec:Introduction}


\section{Dated Log}
\label{sec:Dated_Log}

\subsection{2018 June 6}
\label{subsec:2018_June_6}

\begin{meetingnotes*}
	First off, the main introduction to the scope of the project was introduced.

	Light scattered off of dust is polarized preferentially towards the plane shared by its longest side. In this case, dust is modeled as an oblate shaped particle of varying size.

	Thus, the alignment of the polarization of light coming from a dust based object can reveal information about its rotation. There are two different influences towards the orientation of the dust. 

	Magnetic fields in the dust align the particles of the dust such that the plane shared by its longest side is perpendicular to the B-field. Consider the longest axis of a grain a dust to be the y-axis; the B-field would be in the positive or negative z-axis direction (the specifics towards the exact orientation of the B-field is beyond the scope).

	Next, there are two similar tasks to be completed.
	\begin{enumerate}
		\item Create a function that generates a noisy (non-smooth) gaussian function given some input parameters. The gaussian function should be fully modifiable. 
		\item Create the reverse function: a function that, when given data points from a gaussian, can fit a gaussian function to it.
	\end{enumerate}
\end{meetingnotes*}

Both of these functions have been completed as of this date, written in Python. 

In order to share the functions and code work done with the Mentor, a Github repository has been created. It can be found at \url{https://github.com/psmd-iberutaru/Akamai_Internship}.





\end{document}